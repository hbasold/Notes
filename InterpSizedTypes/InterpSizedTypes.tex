\documentclass[runningheads,envcountsame,envcountsect,orivec]{llncs}
\usepackage[utf8]{inputenc}
\usepackage[english]{babel}
\usepackage{amssymb,amsmath}
\usepackage{stmaryrd,colonequals}
\usepackage{mathtools}
\usepackage{scalerel}
\usepackage[f]{esvect}
%\usepackage{bbm}
\usepackage{xargs}
\usepackage{xspace}
\usepackage{calc}
\usepackage{tikz}
\usepackage{tikz-cd}
\usepackage{array}
\usepackage{etoolbox}
\usepackage{enumitem}
\usepackage{bussproofs}
\usepackage{listings}

%%% Include definitions etc. ...
\usepackage{local-macros}
% ... and "set" options

%%% development packages %%%%
%\usepackage{xcolor}
%\usepackage{changebar}
\usepackage[notref,notcite]{showkeys}
%\usepackage{hyperref}
\usepackage{todonotes}
\usepackage{gitinfo}

%%% package options %%%%
\usetikzlibrary{circuits.logic.CDH,calc,automata,shapes,chains}
% Show only numbers on equations which are referenced
\mathtoolsset{showonlyrefs}

% From:
% https://code.google.com/p/decidr/wiki/LatexLLNCSTableOfContents
% make a proper TOC despite llncs
\setcounter{tocdepth}{2}
\makeatletter
\renewcommand*\l@author[2]{}
\renewcommand*\l@title[2]{}
\makeatletter

\title{Interpretation of Sized Types using Fibrations}
%\author{Henning Basold\inst{1,2}}
%\institute{Radboud University Nijmegen \and CWI Amsterdam}

%\authorrunning{H. Basold}

\begin{document}
\maketitle
\let\thefootnote\relax
\footnotetext{Base revision~\gitAbbrevHash, \gitReferences~from~\gitAuthorDate}

\section{Dependent Fixed Points}
Before getting into the interpretation of dependent sized types, let us first
consider fixed points of dependent types.
Here is an example of a fixed point type and its interpretation over $\SetC$.
\begin{example}
  Assume that for each closed type $\tau$, there is a type
  $n:\NatT \vdash \VecT \tau \, n$, the vectors of size $n$ over $\tau$.
  The we wish to interpret the following judgement
  \begin{equation*}
    n : \NatT \vdash \nu X. (\VecT \tau \, n) \times X.
  \end{equation*}
  The inner type gives rise to a functor $\Endo{G}{\slice{\SetC}{\N}}$ by
  sending $p : \sigma \to \N$ to
  $p \circ \pi_2 : \overline{\VecT \tau} \times_{\N} \sigma \to \N$ given by
  the pullback
  \begin{equation*}
    \begin{tikzcd}
      \overline{\VecT \tau} \times_{\N} \sigma \rar{\pi_1} \dar{\pi_2}
        & \overline{\VecT \tau} \dar{} \\
      \sigma \rar{p} & \N
    \end{tikzcd}
  \end{equation*}
  and a morphism $f : \arrObj{}{\sigma}{\N} \to \arrObj{}{\sigma'}{\N}$ to
  $\id \times_{\N} f$, using the pullback property, i.e.,
  \begin{equation*}
    \begin{tikzcd}
      \overline{\VecT \tau} \times_{\N} \sigma
        \arrow{rr}{\pi_1} \arrow{dd}{\pi_2} \arrow[dashed]{dr}{id \times_{\N} f}
      &
      & \overline{\VecT \tau}
        \dar{\id} \\
      & \overline{\VecT \tau} \times_{\N} \sigma' \rar{\pi_1} \dar{\pi_2}
        & \overline{\VecT \tau} \dar{} \\
      \sigma \rar{f} \arrow[bend right=15]{rr}[swap]{p}
      & \sigma' \rar{p'}
      & \N
    \end{tikzcd}
  \end{equation*}
  Then the fixed point type is interpreted as the final coalgebra of $G$.

  We can actually construct this fixed point by means of the final chain of $G$.
  Since $\arrObj{\id}{\N}{\N}$ is the final object in $\slice{\SetC}{\N}$,
  we construct the chain
  \begin{equation*}
    \begin{tikzcd}
      \arrObjAlt{\id}{\N}{\N}
      & \lar[swap]{!} \arrObjAlt{q}{\overline{\VecT \tau}}{\N}
      & \lar[swap]{G(!)} \arrObjAlt{q_2}{A_2}{\N}
      & \lar[swap]{G^2(!)} \arrObjAlt{q_3}{A_3}{\N}
      & \lar[swap]{G^3(!)} \dotsm
    \end{tikzcd}
  \end{equation*}
  where $A_k = \overline{\VecT \tau}^{\times_{\N}k}$ is the $k$-fold product of
  $\overline{\VecT \tau}$ in $\slice{\SetC}{\N}$ with itself.
  It is then rather easy to see, that the limit of this sequence is given by
  \begin{equation*}
    \lim G^n(!) \cong \coprod_{n \in \N} \Str{(\overline{\tau}^n)},
  \end{equation*}
  that is, a coproduct of streams, where each entry in the stream has the
  same vector size.
\end{example}

Note, that the important property making this work, is that $G$ actually is
a functor on the slice category, thus preserving the context.

\section{Size Environments are Fibred over Size Contexts}
\label{sec:size-env}

Let $\SVar$ be a set of size variable, denoted $i, j$, and $(S, \leq)$ a
well-ordered set, the elements of which we denote by $\alpha, \beta$.
A \emph{size context} $\Delta$ is a list of size variables $i_1, \dotsc, i_n$.
We order contexts by extension, that is, $\Delta \sqsubseteq \Delta'$ iff
$\Delta' = \Delta, i_1', \dotsc, i_k'$.
Using this ordering, we have a category $\SCtx$.

From contexts, we get \emph{size environments} as pairs of contexts $\Delta$
and partial maps $\sigma : \parr{\SVar}{S}$ with $\dom \sigma = \Delta$.
We order size environments lexicographically, that is,
$(\Delta, \sigma) \leq (\Delta', \sigma')$ iff $\Delta \sqsubseteq \Delta'$,
$\Delta = i_1, \dotsc, i_n$, and there is an $m$, s.t.
$\sigma(i_k) = \sigma'(i_k)$ for all $k < m$ and
$\sigma(i_m) \leq \sigma'(i_m)$.
The partial order gives again rise to a category of environments, denoted by
$\SEnv$.

By definition, $(\Delta, \sigma) \leq (\Delta', \sigma')$ implies
$\Delta \sqsubseteq \Delta'$ and so the map $c : \SEnv \to \SCtx$, given by
projecting on the context, is monontone, hence a functor.
Moreover, for all contexts $\Delta \sqsubseteq \Delta'$ and environments
$(\Delta', \sigma')$, we get an environment $(\Delta, \sigma)$, which is
lexicographically smaller than $(\Delta', \sigma')$, by restriction:
$\sigma = \restr{\sigma'}{\Delta}$.
Since all other environments $(\Delta, \tau)$ must be lexicograpically smaller
than $(\Delta, \sigma)$, we actually have that $\sigma$ is the
\emph{cartesian lifting} of $\Delta'$ along $\Delta \sqsubseteq \Delta'$.
Therefore, $c : \SEnv \to \SCtx$ is a fibration.


\section{Interpretation of Fixed Point Approximations}
\label{sec:fp-approx}

In this we section, we expose the structure necessary to interpret sized types,
that is, we approximate fixed points of functors and organise these
approximations by their approximation depth.

To this end, we will work in a category $\Cat{U}$, which is to thought of as
our type universe.
This category has to be total over $\SEnv$, that is, we require to be given
a fibration $\arrObj{p}{\Cat{U}}{\SEnv}$.
Moreover, since we want to interpret products, function types and dependent
types, $\Cat{U}$ should be fibrewise cartesian closed and have fibrewise left
and right adjoints $\Sigma_A \dashv \Delta_A \dashv \Pi_A$ to the diagonal
$\Delta_{A} : \Cat{U}_\sigma \to \slice{\Cat{U}_\sigma}{A}$ for every
environment $\sigma$ and $A : \Cat{U}$.
These are the dependent sum and product, as usual.

An endofunctor on the fibration $\arrObj{p}{\Cat{U}}{\SEnv}$, i.e., a map of
fibrations, is an endofunctor $F$ on $\Cat{U}$ with $p \circ F = p$.
This means, that for every size environment $\sigma$ and type $A : \Cat{U}_A$
above $\sigma$, its image $F A$ must be over $\sigma$.
Since sizes shall represent fixed point approximations, this is exactly what
we want.
We will need the composed fibration $r = c \circ p : \Cat{U} \to \SCtx$,
mapping a type to its size context.

Approximations of the least and greatest fixed point of $F$ are given as
usual by constructing a diagram from smaller approximations and the taking
the colimit or limit, respectively.
To this end, let $\sigma : \SEnv$ a size environment for context $\Delta$,
$i \in \Delta$ and $\alpha = \sigma(i)$.
The semantics of $\mu^iF$ are given by the fibred colimit
$\fpSem{\mu^i F}{\sigma} = \colim D^\sigma_F$ for the following diagram
\begin{align*}
  & D^\sigma_F : \arrObj{q}{S^{< \alpha}}{\SCtx} \to \arrObj{r}{\Cat{U}}{\SEnv} \\
  & D^\sigma_F (\beta) = F(\fpSem{\mu^i F}{\sigma[i \mapsto \beta]}) \\
  & D^\sigma_F (\beta \leq \beta') =
    f : \fpSem{\mu^i F}{\sigma[i \mapsto \beta]}
        \to \fpSem{\mu^i F}{\sigma[i \mapsto \beta']}
\end{align*}
where
\begin{itemize}
\item $q(\beta) = \Delta$ is the constant fibration and
\item $f$ is the given as the unique map from the colimit property, since the
  cocone over $D^{\sigma[i \mapsto \beta']}_F$ from
  $\fpSem{\mu^i F}{\sigma[i \mapsto \beta']}$ restricts to a cocone over
  $D^{\sigma[i \mapsto \beta]}_F$.
\end{itemize}
That $\fpSem{\mu^iF}{\sigma}$ is given as a fibred (over $r$) colimit,
guarantees that the size \emph{context} is not changed in the process.
This is where the fibration $\arrObj{c}{\SEnv}{\SCtx}$ becomes important.
We also note that this process is well-defined, since $S$ is well-ordered,
hence $S^{< \alpha}$ has a minimum, say $\alpha_0$.
Then the colimit of $D^{\sigma[i \mapsto \alpha_0]}_F$ is an initial object in
$\Cat{U}_\Delta$.

Analogously, we can approximate greatest fixed points as limits of a similar
diagram, as follows.
Let again $\sigma : \SEnv$ be a size environment for context $\Delta$,
$i \in \Delta$ and $\alpha = \sigma(i)$.
The semantics of the greatest fixed point is given by the fibred limit
$\fpSem{\mu^iF}{\sigma} = \lim \overline{D}^\sigma_F$ with
\begin{align*}
  & \overline{D}^\sigma_F : \arrObj{q}{\op{(S^{< \alpha})}}{\SCtx}
                           \to \arrObj{r}{\Cat{U}}{\SEnv} \\
  & \overline{D}^\sigma_F(\beta) = F(\fpSem{\nu^i F}{\sigma[i \mapsto \beta]}) \\
  & \overline{D}^\sigma_F(\beta \leq \beta') =
    f : \fpSem{\nu^i F}{\sigma[i \mapsto \beta']}
        \to \fpSem{\nu^i F}{\sigma[i \mapsto \beta]}
\end{align*}
where $q$ is again the constant fibration and $f$ is given by the limit
property as above.

From the (co)limit properties, we also get necessary constructors/destructors:
for all $\beta < \alpha$, we have
\begin{align*}
  & \mathrm{in}^{\alpha}_{\beta} : F\left(\mu^{\beta}\right) \to \mu^{\alpha} \\
  & \mathrm{out}^{\alpha}_{\beta} : \nu^{\alpha} \to F\left(\nu^{\beta}\right).
\end{align*}
This is exactly the structure coming with sized types.

\todo[inline]{
If we can interalise the quantification over $\beta$, we could actually get
$\mathrm{in}^{\alpha} :
  \left(\exists \beta < \alpha .F\left(\mu^{\beta}\right)\right)
  \to \mu^{\alpha}$
and
$\mathrm{out}^{\alpha} : \nu^{\alpha}
  \to \left(\forall \beta < \alpha .F\left(\nu^{\beta}\right)\right)$.
% \begin{align*}
%   & \mathrm{in}^{\alpha} :
%     \left(\exists \beta < \alpha .F\left(\mu^{\beta}\right)\right)
%     \to \mu^{\alpha} \\
%   & \mathrm{out}^{\alpha}_{\beta} : \nu^{\alpha}
%     \to \left(\forall \beta < \alpha .F\left(\nu^{\beta}\right)\right).
% \end{align*}
So far, this is not possible, since the usual definition of quantification
in such a fibred setting requires finite products in $\SEnv$.
In the current setup, $\sigma_1 \times \sigma_2$ is the least environment,
contained lexicographically in, both, $\sigma_1$ and $\sigma_2$.
Then quantification would be a rather weird concept in this setting.
I have to think about this.
}

\section{Example}
Let $\Cat{U} = \SetC \times \SEnv$ and $p = \pi_2$.
This is rather trivially a fibration and probably our main example.
Since for every environment $\sigma$, we have $\Cat{U}_\sigma \cong \SetC$, it
is clear that $\Cat{U}$ is fibrewise cartesian closed and has depent sums and
products.

A functor $\Endo{F}{\Cat{U}}$ is a given by a family
$\{\Endo{G_\sigma}{\SetC}\}_{\sigma : \SEnv}$ of functors.
Moreover, for contexts $\Delta$, the fibre over it with respect to $c \circ p$,
is just $\Cat{U}_\Delta \cong \SetC \times S^\Delta$, where $S^\Delta$ is the
collections of environments over the context $\Delta$.
The diagram $D^\sigma_F : S^{< \alpha} \to \Cat{U}_\Delta$ is then given by
\begin{equation*}
  D^\sigma_F(\beta) = (G_{\sigma'}(\mu^{\sigma'}F), \sigma'),
  \quad \sigma' = \sigma[i \mapsto \beta]
\end{equation*}
and its colimit is
\begin{equation*}
  \mu^{\sigma}F = \colim D^\sigma_F
  = (\colim \, (\pi_1 \circ D^{\sigma}_F), \sigma)
\end{equation*}
where the second colimit is taken in $\SetC$.


\section{Universes}
In the notes, Andreas also mentions $\mathrm{Fun} \, A$, the collection of
endofunctors on $A$.
Since we do not have these internalised into the fibration $p$, the mentioned
monotonicity theorem does not play a role here.
I would rather require a hierarchy of such fibrations, where the functor
category $\IntHom{p}{p}$ lives one level up.

\section{Excursion: Interpretation of Dependent Types}
We give here a naive interpretation of a simple dependent type theory over an
LCCC.
There are some subtleties related to substitution, see
\cite{Jacobs-CatLog, Hofmann-LCCC}.

Assume we are given a signature of basic dependent types together with an
interpretation over $\Cat{U}$.
A type $F$ consists of the following data.
\begin{itemize}
\item a rule for introduction
  \begin{equation*}
    \AxiomC{$\Gamma \vdash A_1 : \ast, \dotsc, A_n : \ast$}
    \AxiomC{$\Gamma \vdash x_1 : B_1, \dotsc, x_m : B_m$}
    \BinaryInfC{$\Gamma \vdash F(A_1, \dotsc, A_n, x_1, \dotsc, x_m) : \ast$}
    \DisplayProof
  \end{equation*}
\item an object $\overline{F}(\alpha_1, \dotsc, \alpha_n) : \Cat{U}$
  for the interpretation $\alpha_i : \overline{A}_i \to \overline{\Gamma}$ of
  $A_i$.
\item a map
  \begin{equation*}
    \Phi_F^{\alpha_1, \dotsc, \alpha_n, \beta_1, \dotsc, \beta_m} :
      \overline{F} \to \overline{\Gamma}
  \end{equation*}
  for each $\alpha_i : \overline{A}_i \to \overline{\Gamma}$
  and $\beta_j : \overline{B_j} \to \overline{\Gamma}$.
  Note that $\Phi_F^{\vec{\alpha},\vec{\beta}}$ is an object in
  $\slice{\Cat{U}}{\,\overline{\Gamma}}$.
\end{itemize}

\begin{example}
  The first example, the natural numbers as a base types, is rather trivial and
  does not require any parameters.
  The introduction rule is
  \begin{equation*}
    \AxiomC{}
    \UnaryInfC{$\Gamma \vdash \NatT : \ast$}
    \DisplayProof
  \end{equation*}
  and the interpretation (over $\SetC$) is, of course, given by
  $\overline{\NatT} = \N$.
  Finally, the context dependency is again trivial, since there is no such
  dependency:
  \begin{equation*}
    \Phi_{\NatT} = \; !_{\overline{\Gamma}}^\ast \, ( !_{\N})
      : \N \to \overline{\Gamma}
  \end{equation*}
  where $!_X : X \to 1$ is the unique arrow into the final object $1$ and
  $!_{\overline{\Gamma}}^\ast :
  \slice{\Cat{U}}{1} \to \slice{\Cat{U}}{\, \overline{\Gamma}}$
  is the reindexing functor.
\end{example}

The second example is less trivial.
\begin{example}
  We define a basic dependent type $\VecT A \, n$ of vectors of type $A$ and
  length $n : \NatT$.
  The introduction rule is
  \begin{equation*}
    \AxiomC{$\Gamma \vdash A : \ast$}
    \AxiomC{$\Gamma \vdash n : \NatT$}
    \BinaryInfC{$\Gamma \vdash \VecT A \; n : \ast$}
    \DisplayProof
  \end{equation*}
  as expected.
  Since the type $A$ can have itself dependencies, the interpretation of $\VecT$
  is a bit tricky.

  Let us first take a look at the special case that $\vdash A : \ast$.
  Then we can just put
  \begin{equation*}
    \overline{\VecT}(\alpha) = \coprod_{n \in \N} \overline{A}^n,
  \end{equation*}
  using that $\alpha : \overline{A} \to 1$, hence $\alpha$ having only one
  fibre, namely $\overline{A}$.
  Now, let $\pi = \termSem{\Gamma \vdash n : \NatT} : \overline{\Gamma} \to \N$
  be the projection on the variable $n$, then the indexing
  for $\VecT$ is given by
  \begin{equation*}
    \Phi_{\VecT}^{\alpha,\beta} = \pi^{\ast}(p)
      : \overline{\VecT}(\alpha) \to \overline{\Gamma}
  \end{equation*}
  where $p : \coprod_{n \in \N} \overline{A}^n \to \N$ is the evident projection
  from the coproduct on its index set.
  Note that we ignore $\beta : \N \to \overline{\Gamma}$ since the type $\NatT$
  does not have any dependencies.

  Next, we are going to tackle the case where $A$ can depend on variables in
  the context $\Gamma$, hence we have
  $\alpha : \overline{A} \to \overline{\Gamma}$.
  The idea is now that each entry in a vector must come from the \emph{same}
  fibre of $\alpha$, thus we put
  \begin{equation*}
    \overline{\VecT}(\alpha) =
      \coprod_{n \in \N}
      \coprod_{x \in \overline{\Gamma}}
      \left( \alpha^{-1}(x) \right)^n.
  \end{equation*}
  The index projection
  $\Phi_{\VecT}^{\alpha,\beta} : \VecT (\alpha) \to \overline{\Gamma}$ is
  given by
  \begin{equation*}
    \prodArr{\pi,id_{\overline{\Gamma}}}^\ast
    \left(p :  \overline{\VecT}(\alpha) \to \N \times \overline{\Gamma}\right)
  \end{equation*}
  where we used the reindexing functor
  \begin{equation*}
    \prodArr{\pi,\id_{\overline{\Gamma}}}^{\ast} :
      \slice{\Cat{U}}{\N \times \overline{\Gamma}} \to
      \slice{\Cat{U}}{\overline{\Gamma}}
  \end{equation*}
  coming from the pairing
  $\prodArr{\pi,\id_{\overline{\Gamma}}} :
  \overline{\Gamma} \to \N \times \overline{\Gamma}$
  and $p$ being again the evident projection on the coproduct indices.

  To fully understand this example, let us take a look at a concrete instance.
  We interpret the judgement
  $n : \NatT, m : \NatT \vdash \VecT \; (\VecT A \; n) \; m$ where $A$ is some
  closed type.
  Then the interpretation of $n : \NatT \vdash \VecT A \; n$ is
  $\Phi_{\VecT} : \coprod_{n \in \N} \overline{A}^n \to \overline{\Gamma}$
  and $\Phi_{\VecT}^{-1}(x) \cong \overline{A}$ for all
  $x \in \overline{\Gamma}$.
  Thus we have
  \begin{equation*}
    \overline{\VecT}(\Phi_{\VecT}) =
      \coprod_{m \in \N}
      \coprod_{x \in \overline{\Gamma}}
      \left(\Phi_{\VecT}^{-1}(x) \right)^m
      \cong
      \coprod_{m \in \N} \coprod_{n \in \N} \overline{A}^n.
  \end{equation*}
\end{example}


%\bibliographystyle{abbrv}
%\bibliography{TypedBDE}

\end{document}

%%% Local Variables:
%%% ispell-local-dictionary: "british"
%%% mode: latex
%%% End:
